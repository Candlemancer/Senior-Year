\documentclass{article}
\usepackage{amsmath}
\usepackage{amssymb}
\usepackage{enumerate}
\title{Homework 4 \\ Section 2.1}
\author{Jonathan Petersen \\ A01236750}
\date{February 1st, 2016}
\begin{document}
	\maketitle
	\hrule
	\begin{enumerate}
		% ======================================================================================== %
		\item [7.]	\textbf{If $a \in \mathbb{Z}$, prove that $a^2$ is not congruent to 2 modulo 4
					or to 3 modulo 4.}

			By Corollorary 2.5, we see that for every $a \in \mathbb{Z}$, either $a \equiv_4 [0]_4$,
			$a \equiv_4 [1]_4$, $a \equiv_4 [2]_4$, or $a \equiv_4 [3]_4$. Further, by the reflexive
			property of congruence, $a \equiv_4 [a]_4$, which implies that $[a]_4 \in \lbrace [0]_4,
		 	[1]_4, [2]_4, [3]_4 \rbrace$.

		 	Finally, we can see by Theorem 2.4 that since $a \equiv_4 a$,

		 		\begin{align*}
		 			a * a & \equiv_4 a * a \\
		 			a^2 & \equiv_4 [a] * [a] \\
		 			a^2 & \equiv_4 [a]^2 \\
		 		\end{align*}

		 	and therefore 

		 		\begin{align*}
		 			[a]^2 \in \lbrace {[0]_4}^2, {[1]_4}^2, {[2]_4}^2, {[3]_4}^2 \rbrace \\
		 			[a]^2 \in \lbrace [0]_4, [1]_4, [4]_4, [9]_4 \rbrace \\ 
		 			[a]^2 \in \lbrace [0]_4, [1]_4, [0]_4, [1]_4 \rbrace \\
		 			[a]^2 \in \lbrace [0]_4, [1]_4 \rbrace \\
		 		\end{align*}

		 	which implies that $[a]^2 \not\equiv_4 2$ and $[a]^2 \not\equiv_4 3$ $_{\blacksquare}$

		% ======================================================================================== %
		\item [8.]	\textbf{Prove that every odd integer is congruent to 1 modulo 4 or 3 modulo 4.}

			Given any odd integer, we know that the integer may be expressed as $2i + 1$ with $i \in
			\mathbb{Z}$. We also know by the definition of equivalence class that 

				\begin{equation*}
					[2i + 1]_4 = {x \in \mathbb{Z} s.t. 4 | x - (2i + 1)}	
				\end{equation*}

			If we now examine the statement $4 | x - (2i + 1)$, we can see that since $x - (2i + 1)$
			must be divisible by $4$, $x - (2i + 1)$ must be even. Therefore, by the properties of 
			subtraction on even and odd numbers, $x - (2i + 1)$ could only be odd when $x$ is odd.

			Finally, we also know from Corollary 2.5 that 
			
				\begin{equation*}
					[x]_4 \in \lbrace [0]_4, [1]_4, [2]_4, [3]_4 \rbrace
				\end{equation*}

			but since $x$ must be odd we are only left with the possibilities
			
				\begin{equation*}
					[x]_4 \in \lbrace [1]_4, [3]_4 \rbrace
				\end{equation*}

			Substituting for $x$ we find that

				\begin{equation}
					[2i + 1]_4 \in \lbrace [1]_4, [3]_4 \rbrace
				\end{equation} 

			and indeed, if $i = 0$ then $2i + 1$ must be in $[1]_4$, and if $i = 1$ then it must be 
			in $[3]_4$. Therefore, we know that $2i + 1$ must be congruent to 1 modulo 4 or 3 modulo
			4, and that both cases exist $_{\blacksquare}$

		% ======================================================================================== %
		\item [17.]	\textbf{Prove that $10^n \equiv_{11} (-1)^n$ for $n > 0$, $n \in \mathbb{Z}$.}

			Since $11 = 10 - (-1)$, it is clear that 

				\begin{align*}
					11 | (10 & - (-1)) \\
					10 & \equiv_{11} -1 \\
				\end{align*}

			and so by Theorem 2.2, we can see that $10^n \equiv_{11} (-1)^n$ $_{\blacksquare}$

		% ======================================================================================== %
		\item [21.]	
			\begin{enumerate}
				\item [a.]	\textbf{Show that $10^n \equiv_9 1^n$ for $n > 0$, $n \in 
							\mathbb{Z}$.}

					Similar to the logic in problem 17, we see that:

						\begin{align*}
							9 | (10 & - 1) \\
							10 & \equiv_9 1 \\
						\end{align*}

					So by Theorem 2.2 we find that $10^n \equiv_9 1^n \equiv_9 1$ 
					$_{\blacksquare}$

				\item [b.]	\textbf{Prove that every integer is congruent to the sum of its digits
							mod 9.}		

					It is clear that $9 | 10 - 1$. Now suppose that $9 | 10^n - 1$. Then 

						\begin{align*}
							10^{n + 1} - 1 & = 10(10^n) - 1 \\
								&= (9(10^n) + 10^n) - 1 \\
								&= 9(10^n) + 10^n - 1
						\end{align*}

					Therefore $9 | 10^{n+1} - 1$, and by induction it follows that $9 | 10^n -1$.
					Now consider an arbitrary integer $a$ expressed as

					\begin{equation*}
						a = 10^nd_n + 10^{n-1}d_{n-1} + ... + 10^2d_2 + 10^1d_1 + 10^0d_0
					\end{equation*}

					As we showed above, each of the terms composing $a$ are divisible by 9, which 
					by theorem 2.2 means that they are all congruent. Furthermore, their sums are
					all congruent, and therefore $a \equiv_9 10^nd_n + 10^{n-1}d_{n-1} + ... + 
					10^2d_2 + 10^1d_1 + 10^0d_0$ $_{\blacksquare}$

			\end{enumerate}
		% ======================================================================================== %
		\item [22.]	
			\begin{enumerate}
				\item [a.]	\textbf{Give an example to show that the following statement is false:
							If $ab \equiv_n ac$, and $a \not\equiv_n 0$, then $b \equiv_n c$.}

					Let $a = 2, b = 2, c = 4, n = 4$. Then $ab = 4$, $ac = 8$, and $ab \equiv_4 ac$.
					Also, $a \not\equiv_4 0$, but $b = 2 \not\equiv_4 = 4 = c$ $_{\blacksquare}$

				\item [b.]	\textbf{Prove that the above statement is true when $gcd(a, n) = 1$.}

					Given the statement $ab \equiv_n ac$, we see that this may only be true if 

						\begin{align*}
							n & | ab - ac \\
							n & | a(b - c) \\
							n & | ak \qquad k = (b - c) \therefore k \in \mathbb{Z} \\
						\end{align*}

					By the properties of division, we know this may be true only if $n | a$ or $n |
					k$. It then follows that if the $gcd(a, n) = 1$ that $n$ does not divide $a$, 
					and therefore

						\begin{align*}
							n & | k \\
							n & | b - c \\
							b \equiv_n c
						\end{align*}

					by definition $_{\blacksquare}$

			\end{enumerate}
	\end{enumerate}
\end{document}

