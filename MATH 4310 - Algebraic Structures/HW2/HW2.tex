\documentclass{article}
\usepackage{amsmath}
\usepackage{amssymb}
\usepackage{enumerate}
\title{Homework 2 \\ Section 1.2}
\author{Jonathan Petersen \\ A01236750}
\date{January 24th, 2016}
\begin{document}
	\maketitle
	\hrule
	\begin{enumerate}
		% ==========================================================================================
		\item[3.]	\textbf{Prove that if $a | b$ and $b | c$, then $a | c$.}

			Since $a | b$, we know that
			\begin{equation*}
				b = am \qquad m \in \mathbb{Z}
			\end{equation*}

			By a similar argument,
			\begin{equation*}
				c = bn \qquad n \in \mathbb{Z}
			\end{equation*}

			Substituting $b = am$ in this equation, we find that
			\begin{align*}
				c &= amn \\
				c &= ak \qquad k = mn
			\end{align*}

			Since $m, n \in \mathbb{Z}$, and since multiplication is closed over $\mathbb{Z}$,
			$k \in \mathbb{Z}$. Therefore it must be that, $a | c$ $_{\blacksquare}$

		% ==========================================================================================
		\item[8.]	\textbf{Prove that $gcd(n, n+1) = 1$ for every integer $n$.}

			Let $x$ be the GCD of $n, n+1$. By the Linear Combination property, we know that $x$ 
			can be written as:

			\begin{align*}
				x &= nu + (n+1)v \qquad u, v \in \mathbb{Z} \\
				x &= nu + nv + v
			\end{align*}

			If we then let $u = -1$ and $v = 1$, we find that:

			\begin{align*}
				x &= -n + n + 1 \\
				x &= 1 \\
				1 &= nu + (n+1)v
			\end{align*}

			Or, in other words, that $gcd(n, n+1) | 1$. As discussed in class, this shows that the 
			GCD of $n, n+1$ must be 1, as 1 is the only divisor of 1 $_{\blacksquare}$


		% ==========================================================================================
		\item[11.]	\textbf{If $n \in \mathbb{Z}$, what are the possible values of} 
		\begin{enumerate}
			\item[a.]	$gcd(n, n+2)$ \\
				1, 2
			\item[b.]	$gcd(n, n+6)$ \\
				1, 2, 3
		\end{enumerate}

		% ==========================================================================================
		\item[14.]	\textbf{Find the smallest positive integer in the given set.}
		\begin{enumerate}
			\item[a.]	$\lbrace 6u + 15v \vert u, v \in \mathbb{Z} \rbrace$. 

				Since $3 | 6$ and $3 | 15$, we see that
				
				\begin{equation*}
					6u + 15v = 3(2u + 5v)
				\end{equation*}

				Furthermore, to find the smallest positve integer in the set it is obvous that we 
				must make $2u + 5v$ as small as possible, without becoming zero. The smallest 
				possible positive integer would be $2u + 5v = 1$, and indeed if $u = -7, v = 3$ we
				find that

				\begin{align*}
					3(2u + 5v) &= 3(2(-7) + 5(3)) \\
							   &= 3(-14 + 15) \\
							   &= 3(1) 
				\end{align*}

				Which must be the smallest value as shown above.

			\item[b.]	$\lbrace 12r + 17s \vert r, s \in \mathbb{Z} \rbrace$. 

				By similar reasoning, if we can find $r, s$ such that $12r + 17s = 1$, it must also 
				be the smallest positive integer. And indeed, if we set $r = -7, s = 5$ we see that

				\begin{align*}
					12r + 17s &= 12(-7) + 17(5) \\
							  &= -84 + 85 \\
							  &= 1
				\end{align*}

				Which must mean that 1 is the smallest positive integer in the set.
		\end{enumerate}

		% ==========================================================================================
		\item[22.]	\textbf{If $gcd(a, c) = 1$ and $gcd(b, c) = 1$, prove that $gcd(ab, c) = 1$.}

		Given that $gcd(a, c) = 1$ and $gcd(b, c) = 1$, we know by the Linear Combination property
		that 

		\begin{align*}
			1 &= au_1 + cv_1 \\
			1 &= bu_2 + cv_2
		\end{align*}

		and it then follows that

		\begin{equation*}
			a = abu_2 + acv_2
		\end{equation*}

		which, by substitution, means that

		\begin{align*}
			1 &= (abu_2 + acv_2) + cv_1 \\
			  &= abu_2 + c(av_2 + v_1) \\
			  &= abu_3 + cv_3 \qquad \qquad u_3 = u_2, v_3 = av_2 + v_1
		\end{align*}

		By the converse of the Linear Combination property in the special case when considering a 
		GCD of 1, this shows that if $gcd(a, c) = 1$ and $gcd(b, c) = 1$, then $gcd(ab, c) = 1$
		$_{\blacksquare}$

		% ==========================================================================================
		\item[20.] 	Extra Credit: Prove that $gcd(a, b) = gcd(a, b + at)$ for every $t \in 
					\mathbb{Z}$.

	\end{enumerate}
\end{document}

