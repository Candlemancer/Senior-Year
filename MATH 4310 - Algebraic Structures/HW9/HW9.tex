\documentclass{article}
\usepackage{amsmath}
\usepackage{amssymb}
\usepackage{amsfonts}
\usepackage{enumerate}
\usepackage{centernot}
\newcommand{\overbar}[1]{\mkern 1.5mu\overline{\mkern-1.5mu#1\mkern-1.5mu}\mkern 1.5mu}
\title{Homework 9 \\ Section 3.3}
\author{Jonathan Petersen \\ A01236750}
\date{March 21st, 2016}
\begin{document}
	\maketitle
	\hrule 
	\vspace{5mm}
	\begin{enumerate}
		% ======================================================================================== %
		\item [3.] \textbf{Show that $f:R \rightarrow R^*$ with $R^* = \lbrace (r, r) | (r, r) \in 
						   R \times R, r \in R \rbrace$ given by $f(a) = (a, a) $is an isomorphism.}

			To show that $f$ is an isomorphism, we must show that it is a bijective homomorphism, or
			that it meets the following four criteria:
			\begin{enumerate}
				\item $f(a + b) = f(a) + f(b)$ ($f$ preserves addition) \\
				\item $f(a * b) = f(a) * f(b)$ ($f$ preserves multiplication) \\
				\item $f(a) = f(b) \implies a = b$ ($f$ is injective) \\
				\item For any $c$ in the range, there exists some $f(a) = c$. ($f$ is surjective)
			\end{enumerate}

			We shall check each criteria individually. First, to show that $f$ preserves addition we
			observe
			\begin{align*}
				f(a + b) & = (a + b, a + b) \qquad \qquad a, b \in R \\
						 & = (a, a) + (b, b) \\
						 & = f(a) + f(b) \\
			\end{align*}

			Therefore, $f$ preserves addition. By a similar fashion, let us check if $f$ preserves
			multiplication.
			\begin{align*}
				f(a * b) & = (a * b, a * b) \qquad \qquad a, b \in R \\
						 & = (a, a) * (b, b) \\
						 & = f(a) * f(b) \\
 			\end{align*}

 			So $f$ preserves multiplicaton and is therefore a homomorphism. To check that $f$ is 
 			injective, consider
 			\begin{align*}
 				f(a) = c \qquad f(b) & = c \qquad \qquad a, b \in R \qquad c \in R^* \\
 				f(a) & = (a, a) \\
 				c & = (a, a) \\
 				f(b) & = (b, b) \\
 				c & = (b, b) \\
 				(a, a) & = (b, b) \\
 				a & = b \\
 			\end{align*}

 			Which shows that $f$ is injective. Finally, we check if $f$ is surjective. Let
 			\begin{align*}
 				(c, c) & \in R^* \\
 				c & \in R \\
 				f(c) & = (c, c) \\
 			\end{align*}

 			Thusly, $f$ meets the four criteria and we may conclude that $f$ is an isomorphism
 			$_{\blacksquare}$

		% ======================================================================================== %
		\item [8.] \textbf{Let $\mathbb{Q}(\sqrt{2}) = \lbrace r + s\sqrt{2} | r, s \in \mathbb{Q}
						   \rbrace$. Prove that $f:\mathbb{Q}(\sqrt{2}) \rightarrow 
						   \mathbb{Q}(\sqrt{2})$ given by $f(r + s\sqrt{2}) = r - s\sqrt{2}$ is an 
						   isomorphism.}

			Following the structure in problem 3, we shall check that this problem also meets the 
			same four criteria. First, to show that $f$ preserves addition we observe
			\begin{align*}
				f(a + b) & = f(a_1 + a_2\sqrt{2} + b_1 + b_2\sqrt{2}) 
										\qquad a,b \in \mathbb{Q}(\sqrt{2}) \\
						 & = f((a_1 + b_1) + (a_2 + b_2)\sqrt{2}) \\
						 & = (a_1 + b_1) - (a_2 + b_2)\sqrt{2} \\
						 & = a_1 + b_1 - a_2\sqrt{2} - b_2\sqrt{2} \\
						 & = a_1 - a_2\sqrt{2} + b_1 - b_2\sqrt{2} \\
						 & = f(a) + f(b) \\
			\end{align*}

			Therefore, $f$ preserves addition. By a similar fashion, let us check if $f$ preserves
			multiplication.
			\begin{align*}
				f(a * b) & = f(a_1 + a_2\sqrt{2} * b_1 + b_2\sqrt{2}) 
									\qquad a,b \in \mathbb{Q}(\sqrt{2}) \\
						 & = f(a_1 * b_1) + (a_1 * b_2\sqrt{2}) + (a_2\sqrt{2} * b_1) + 
						 			(a_2\sqrt{2} * b_2\sqrt{2}) \\
						 & = f((a_1 * b_1 + 2 * a_2 * b_2) + (a_1 * b_2 + a_2 * b_1)\sqrt{2}) \\
						 & = (a_1 * b_1 + 2 * a_2 * b_2) - (a_1 * b_2 + a_2 * b_1)\sqrt{2} \\
						 & = (a_1 * b_1) - (a_2\sqrt{2} * b_1) - (a_1 * b_2\sqrt{2}) + 
						 			(a_2\sqrt{2} * b_2\sqrt{2})\\
						 & = (a_1 - a_2\sqrt{2}) * (b_1 - b_2\sqrt{2}) \\
						 & = f(a) * f(b) \\
 			\end{align*}

 			So $f$ preserves multiplicaton and is therefore a homomorphism. To check that $f$ is 
 			injective, consider
 			\begin{align*}
 				f(a) = c \qquad f(b) & = c \qquad \qquad a, b, c \in \mathbb{Q}(\sqrt{2}) \\
 				f(a) & = a_1 - a_2\sqrt{2} \\
 				c & = a_1 - a_2\sqrt{2} \\
 				f(b) & = b_1 - b_2\sqrt{2} \\
 				c & = b_1 - b_2\sqrt{2} \\
 				a_1 - a_2\sqrt{2} & = b_1 - b_2\sqrt{2} \\
 				a_1 & = b_1 \qquad a_2 = b_2 \\
 			\end{align*}

 			Which shows that $f$ is injective. Finally, we check if $f$ is surjective. Let
 			\begin{align*}
 				c_1 + c_2\sqrt{2} & \in \mathbb{Q}(\sqrt{2}) \\
 				c_1 + c_2\sqrt{2} + 0 - 2c_2\sqrt{2} & \in \mathbb{Q}(\sqrt{2}) \\
 				c_1 - c_2\sqrt{2} & \in \mathbb{Q}(\sqrt{2}) \\
 				f(c_1 - c_2\sqrt{2}) & = c_1 + c_2\sqrt{2} \\
 			\end{align*}

 			Thusly, $f$ meets the four criteria and we may conclude that $f$ is an isomorphism
 			$_{\blacksquare}$

		% ======================================================================================== %
		\item [9.] \textbf{If $f:\mathbb{Z} \rightarrow \mathbb{Z}$ is an isomorphism, prove that 
						   $f$ is the identity map.}

			Recall that $f(0) = 0$ for all rings. Furhtermore, since $f$ is an isomorphism, it 
			preserves addition. Therefore
			\begin{align*}
				f(0) & = 0 \\
				f(1) = f(0 + 1) = f(0) & + f(1) = 0 + 1 = 1 \\
				f(2) = f(1 + 1) = f(1) & + f(1) = 1 + 1 = 2 \\
				f(3) = f(2 + 1) = f(2) & + f(1) = 2 + 1 = 3 \\
				\vdots \\
				f(n) = f(n-1) + f(1) = f(n-1) & + f(1) = n - 1 + 1 = n \\
			\end{align*}

		% ======================================================================================== %
			Teacher Solution:
				Since $1 \in Z$ and $0 \in Z$, we know that the above is true.

		% ======================================================================================== %
		\item [15.] \textbf{Let $f:R \rightarrow S$ be a homomorphism of rings. if $r$ is a zero 
							divisor in $R$, is $f(r)$ a zero divisor in $S$?}			

			Yes. Observe
			\begin{align*}
				r * x & = 0_R \\
				f(r) * f(x) & = f(0_R) \\
							& = 0_S \\ 
			\end{align*}
			Therefore $f(r)$ is a zero divisor of $f(x)$ in $S$ $_{\blacksquare}$

		% ======================================================================================== %
		\item [17.] \textbf{Show that the complex conjugation function $f:\mathbb{C} \rightarrow 
							\mathbb{C}$ given by $f(a + ib) = a - ib$ is a bijection.}

			Let $a + ib$ and $c + id$ be arbitrary elements in $\mathbb{C}$. Then if 
			\begin{align*}
				f(a + ib) = x - yi \qquad \qquad x - yi \in \mathbb{C} \\
				f(c + id) = x - yi \\
				x = a = c & y = b = d \\
				a = c & b = d \\
			\end{align*} 
			Therefore $f$ is injective. Now consider again some arbitrary $x - yi$ in $\mathbb{C}$.
			\begin{align*}
				x - yi = f(x + yi)
			\end{align*}
			And since $x + yi$ must also be in $\mathbb{C}$, $f$ is surjective. Since $f$ is both
			injective and surjective, $f$ is bijective $_{\blacksquare}$

		% ======================================================================================== %
		\item [19.] \textbf{Show that $S = \lbrace 0, 4, 8, 12, 16, 20, 24 \rbrace$ is a subring of
							$\mathbb{Z}_{28}$. Then prove that the map $f:\mathbb{Z}_7 \rightarrow
							S$ given by $f([x]_7) = [8x]_{28}$ is an isomorphism.}

			To show that $S$ is a subring of $\mathbb{Z}_{28}$, we must show that it is closed under
			the subtraction and multiplication operators of $\mathbb{Z}_{28}$. Then, to show that
			$f$ is an isomorphism, we must prove the same four qualities shown in the first problem.

		% ======================================================================================== %
		\item [22.] \textbf{Let $\overbar{\mathbb{Z}}$ denote the ring of integers with $a 
							\oplus b = a + b - 1$ and $a \odot b = ab - (a + b) + 2$ operations.
							Prove that $\overbar{\mathbb{Z}}$ is isomorphic to $\mathbb{Z}$.}

			To show that $\overbar{\mathbb{Z}}$ is isomorphic to $\mathbb{Z}$, we again must show
			that it is homomorphic and bijective. Let us begin by checking preservation of addition:
			\begin{align*}
				f(a \oplus b) & = f(a + b - 1) \\
							  & = f(a) + f(b) + f(-1)
			\end{align*}

			Next, preservation of multiplication:
			\begin{align*}
				f(a \odot b) & = f(ab - (a + b) + 2) \\
							 & = f(ab) - f(a + b) + f(2) \\
							 & = f(a) * f(b) - f(a) - f(b) + f(2) \\
			\end{align*}

			Next, injectivity:
		% ======================================================================================== %
			Teacher Solution:
				Step 1: $f(1) = e$ where $e$ is the identity in the range. 
				Step 2: Find $e$.
				Step 3: Find $f(n)$.
				Now start fresh by proving isomorphism on f.

		% ======================================================================================== %
		\item [27.a.] \textbf{If $g:R \rightarrow S$ and $f:S \rightarrow T$ are homomorphisms, show
							  that $f \circ g:R \rightarrow T$ is a homomorphism.}		

			To show that $f \circ g:R \rightarrow T$ is a homomorphism, we must show that it 
			preserves addition and multiplication from $R \rightarrow T$. That is, that
			\begin{align*}
				f(g(x + y)) & = f(g(x)) + f(g(y)) \qquad x, y \in R \\
				f(g(x * y)) & = f(g(x)) * f(g(y)) \\
			\end{align*}

			Indeed, since $f$ and $g$ are homomorphisms,
			\begin{align*}
				f(g(x + y)) & = f(g(x) + g(y)) \\ 
							& = f(g(x)) + f(g(y)) \\
			\end{align*}
			\begin{align*}
				f(g(x * y)) & = f(g(x) * g(y)) \\ 
							& = f(g(x)) * f(g(y)) \\
			\end{align*}

			So $f \circ g:R \rightarrow T$ is a homomorphism $_{\blacksquare}$

		% ======================================================================================== %
		\item [27.b.] \textbf{If $f$ and $g$ are isomorphisms, show that $f \circ g$ is also an 
							  isomorphism.}		

			As shown in the previous problem, if $f$ and $g$ are homomorphisms, then $f \circ g$ is
			also a homomorphism. Since an isomorphism must be a homomorphism, all that remains is 
			to show that if $f$ and $g$ are isomorphims, that $f \circ g$ is bijective.

			Therefore, let us consider an arbitrary $a, b$ in the domain of $g$, and some $c$ in the
			range of $f$. Then
			\begin{align*}
				f(g(a)) = c & f(g(b)) = c \\
				f(g(a)) = c \qquad f(g(b)) = c & \implies g(a) = g(b) \\
				g(a) = c \qquad g(b) = c & \implies a = b \\
			\end{align*} 
			So $f \circ g$ is injective. To check surjectivity, take $c$ as an arbitrary element in
			the range of $f$. Then since $g$ and $f$ are isomorphic, 
			\begin{align*}
				c \in R(f) & \implies f(b) \\
				f(b) \in R(g) & \implies g(a) \\
				\therefore f(g(a)) = c
			\end{align*}
			And we see that $f \circ g$ is surjective and therefore an isomorphism $_{\blacksquare}$

		% ======================================================================================== %
		\item [35.a.] \textbf{Show that $E$ and $\mathbb{Z}$ are not isomorphic.}

			$\mathbb{Z}$ is a ring with identity, and $E$ is not. Therefore, the two cannot be 
			isomorphic.

		% ======================================================================================== %
		\item [35.b.] \textbf{Show that $\mathbb{R} \times \mathbb{R} \times \mathbb{R} \times 
							  \mathbb{R}$ and $M(\mathbb{R})$ are not isomorphic.}

			$\mathbb{R} \times \mathbb{R} \times \mathbb{R} \times \mathbb{R}$ is a commutative
			ring, and $M(\mathbb{R})$ is not, so they cannot be isomorphic.

		% ======================================================================================== %
		\item [35.c.] \textbf{Show that $\mathbb{Z}_4 \times \mathbb{Z}_{14}$ and $\mathbb{Z}_{16}$ 
							  are not isomorphic.}

			There are $4 * 14 = 56$ elements in $\mathbb{Z}_4 \times \mathbb{Z}_{14}$, but only $16$
			in $\mathbb{Z}_{16}$. Therefore the two cannot be isomorphic.

		% ======================================================================================== %
	\end{enumerate}
\end{document}

