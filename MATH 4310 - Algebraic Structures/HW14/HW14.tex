	\documentclass{article}
\usepackage{amsmath}
\usepackage{amssymb}
\usepackage{amsfonts}
\usepackage{enumerate}
\usepackage{centernot}
\usepackage{polynom}
\newcommand{\overbar}[1]{\mkern 1.5mu\overline{\mkern-1.5mu#1\mkern-1.5mu}\mkern 1.5mu}
\title{Homework 14 \\ Section 5.1, 5.2, 5.3}
\author{Jonathan Petersen \\ A01236750}
\date{April 27th, 2016}
\begin{document}
	\maketitle
	\hrule 
	\vspace{5mm}
	\begin{enumerate}
		% ======================================================================================== %
		\item [5.1.1.b. ] \textbf{Let $f(x)$, $g(x)$, $p(x) \in F[x]$, with $p(x)$ nonzero. 
			Determine whether $f(x) \equiv g(x)$ (mod $p(x)$). Show your work. Let $f(x) = x^4 + x^2
			+ x + 1$, $g(x) = x^4 + x^3 + x^2 + 1$, $p(x) = x^2 + x$, and $F = \mathbb{Q}$.}

			To check if $f(x)$ and $g(x)$ are equivalent modulo $p(x)$, we must check to see if the 
			remainders of $f(x)$ and $g(x)$ are the same up to associates when divided by $p(x)$.
			For $f(x)$, observe that 

				$\polylongdiv{x^4 + x^2+ x + 1}{x^2 + x}$

			and for $g(x)$ that

				$\polylongdiv{x^4 + x^3 + x^2 + 1}{x^2 + x}$

			In other words, 
			\begin{align*}
				f(x) & = (x^2 + x)(x^2 - x + 2) - x \\
				g(x) & = (x^2 + x)(x^2 + 1) - x
			\end{align*}
			and since the remainder in both cases is $-x$, $f(x) \equiv g(x)$ (mod $p(x)$).

		% ======================================================================================== %
		\item [5.1.1.c. ] \textbf{Let $f(x)$, $g(x)$, $p(x) \in F[x]$, with $p(x)$ nonzero. 
			Determine whether $f(x) \equiv g(x)$ (mod $p(x)$). Show your work. Let $f(x) = 3x^5 + 
			4x^4 + 5x^3 - 6x^2 + 5x - 7$, $g(x) = 2x^5 + 6x^4 + x^3 + 2x^2 + 2x - 5$, $p(x) = x^3 -
			x^2 + x - 1$, and $F = \mathbb{R}$.}

			As in the above, observe that for $f(x)$

			$\polylongdiv{3x^5 + 4x^4 + 5x^3 - 6x^2 + 5x - 7}{x^3 - x^2 + x - 1}$

			and for $g(x)$,

			$\polylongdiv{2x^5 + 6x^4 + x^3 + 2x^2 + 2x - 5}{x^3 - x^2 + x - 1}$

			We see that
			\begin{align*}
				f(x) & = (x^3 - x^2 + x - 1)(3x^2 + 7x + 9) + (-x^2 + 3x + 2) \\
				g(x) & = (x^3 - x^2 + x - 1)(2x^2 + 8x + 7) + (3x^2 + 3x + 2)
			\end{align*}
			and since $-x^2 + 3x + 2 \neq 3x^2 + 3x + 2$, $f(x) \not\equiv g(x)$ (mod $p(x)$).

		% ======================================================================================== %
		\item [5.1.3. ] \textbf{How many distinct congruence classes are there modulo $x^3 + x + 1$
			in $\mathbb{Z}_2[x]$? List them.}

			We know from the definition of congruence classes that there must be a distinct
			congruence class for every distinct remainder value when an indeterminate polynomial in
			$\mathbb{Z}_2[x]$ is divided by $x^3 + x + 1$. We also know that the degree of the 
			remainder in such a case must be less than the degree of $x^3 + x + 1$, and as such all
			possible remainders can be represented as a list of all possible polynomials of degree
			2 in $\mathbb{Z}_2[x]$. They are as follows:

			\begin{enumerate}
				\item $[0]x^2 + [0]x + [0] = [0]$
				\item $[0]x^2 + [0]x + [1] = [1]$
				\item $[0]x^2 + [1]x + [0] = x$
				\item $[0]x^2 + [1]x + [1] = x + [1]$
				\item $[1]x^2 + [0]x + [0] = x^2$
				\item $[1]x^2 + [0]x + [1] = x^2 + [1]$
				\item $[1]x^2 + [1]x + [0] = x^2 + x$
				\item $[1]x^2 + [1]x + [1] = x^2 + x + [1]$
			\end{enumerate}

			and as we can see, there are eight possible values. Therefore, there must be eight 
			distinct congruence classes in $\mathbb{Z}_2[x]$ (mod $x^3 + x + 1$).

		% ======================================================================================== %
		\item [5.1.12. ] \textbf{If $f(x)$ is relatively prime to $p(x)$, prove that there is a 
			polynomial $g(x) \in F[x]$ such that $f(x)g(x) \equiv 1_F$ (mod $p(x)$).}

			Since $f(x)$ and $p(x)$ are relatively prime, $f(x)$ must be a unit in $F[x]/<p(x)>$. 
			Furthermore, by Theorem 4.8 there must be polynomials $g(x)$, $h(x)$ such that
			\begin{align*}
				f(x)g(x) + p(x)h(x) & = [1] \\
				f(x)g(x) - [1] & = -p(x)h(x) \\
					& = p(x)(-h(x)) 
			\end{align*}
			and by Theorem 5.3, this implies that 
			\begin{equation*}
				[f(x)g(x)] = [1]
			\end{equation*}
			or rather that $f(x)g(x) \equiv 1_F$ (mod $p(x)$) $_{\blacksquare}$

		% ======================================================================================== %
		\item [5.2.2. ] \textbf{Write out the addition and multiplicaiton tables for 
			$\mathbb{Z}_3[x]/<x^2 + 1>$. Is $\mathbb{Z}_3[x]/<x^2 + 1>$ a field?}

			Since $x^2 + 1$ is irreducible in $\mathbb{Z}_3[x]$, $\mathbb{Z}_3[x]/<x^2 + 1>$ is a 
			field.

			\newpage

		% ======================================================================================== %
		\item [5.2.14.a. ] \textbf{Explain why $[f(x)] = [2x - 3] \in \mathbb{Q}[x]/<x^2 - 2>$ is a
			unit and find its inverse.}

			Since $x^2 - 2$ is irreducible in $\mathbb{Q}[x]$, $\mathbb{Q}[x]/<x^2 - 2>$ is a field,
			and so every nonzero element of $\mathbb{Q}[x]/<x^2 - 2>$ is a unit. Since $[2x - 3]$ is
			a nonzero element of $\mathbb{Q}[x]/<x^2 - 2>$, it must also be a unit. 

			By Theorem 4.8, there must be some $g(x)$, $q(x)$ such that if $p(x) = x^2 - 2$ 
			\begin{equation*}
				f(x)g(x) + p(x)q(x) = [1]
			\end{equation*}
			We also know that $g(x)$ and $q(x)$ must have degree smaller than $p(x)$, namely degree
			one. Therefore, without loss of generality, we can assume that
			\begin{align*}
				f(x)(ax + b) + p(x)(cx + d) & = 1 \\
				(2x - 3)(ax + b) + (x^2 - 2)(cx + d) & = 1 \\
				2ax^2 + 2bx - 3ax - 3b + cx^3 + dx^2 - 2cx - 2d & = 1 \\
				cx^3 + (2a + d)x^2 + (2b - 3a - 2c)x + (-3b - 2d) & = 1
			\end{align*}
			Which, by equality of polynomials, leads to the system of equations
			\begin{align*}
				c & = 0 \\
				2a + d & = 0 \\
				2b - 3a - 2c & = 0 \\
				-3b - 2d & = 1 \\
			\end{align*}
			So therefore
			\begin{align*}
				% c & = 0 \\
				% d & = -2a \\
				% -3b - 2(-2a) = 1
				% -3b + 4a = 1
				% 4a = 1 + 3b
				% a = (1 + 3b) / 4
				% 2b - 3((1 + 3b) / 4) - 0 = 0
				% 2b - 3/4 -9/4b = 0
				% (2 - 2.25)b = 3/4
				% -1/4b = 3/4
				% -b = 3
				% b = -3
				% -3(-3) -2d & = 1
				% 9 - 2d = 1
				% -2d = -8
				% d = 4
				% 4a = 1 + 3(-3)
				% 4a = 1 - 9
				% 4a = -8
				% a = -2
				a = -2 \\
				b = -3 \\
				c = 0 \\ 
				d = 4 
			\end{align*}
			and the inverse of $[f(x)] = [2x - 3]$ is $[g(x)] = [-2x -3]$.

		% ======================================================================================== %
		\item [5.2.14.b. ] \textbf{Explain why $[f(x)] = [x^2 + x + 1] \in \mathbb{Z}_3[x]/
			<x^2 + 1>$ is a unit and find its inverse.}

			Let $p(x) = x^2 + 1$ in $\mathbb{Z}_3[x]$. Since $f(x)$ and $p(x)$ are relatively prime,
			$f(x)$ must be a unit in $\mathbb{Z}_3[x]/<x^2 + 1>$. By the same logic as the previous
			problem, there must be a $g(x) = ax + b$ and $q(x) = cx + d$ such that
			\begin{align*}
				f(x)g(x) + p(x)q(x) = 1 \\
				(x^2 + x + 1)(ax + b) + (x^2 + 1)(cx + d) & = 1 \\
				ax^3 + ax^2 + ax + bx^2 + bx + b + cx^3 + dx^2 + cx + d & = 1 \\
				(a + c)x^3 + (a + b + d)x^2 + (a + b + c)x + (b + d) & = 1
			\end{align*}
			Which, by equality of polynomials, leads to the system of equations
			\begin{align*}
				a + c & = 0 \\
				a + b + d & = 0 \\
				a + b + c & = 0 \\
				b + d = 1 \\
			\end{align*}
			And therefore
			\begin{align*}
				% a & = -c \\
				% d & = c \\
				% b & = 1 - d \\
				% b & = 1 - c \\
				% -c + (1 - c) + c = 0
				% 1 - c = 0
				a = -1 \\
				b = 0 \\
				c = 1 \\ 
				d = 1 \\
			\end{align*}
			so the inverse of $[f(x)]$ is $[g(x)] = [-x]$.

		% ======================================================================================== %
		\item [5.3.1.a. ] \textbf{Determine whether $\mathbb{Z}_3[x]/<x^3 + 2x^2 + x + 1>$ is a 
			field. Justify your answer.}

			$\mathbb{Z}_3[x]/<x^3 + 2x^2 + x + 1>$ is a field if and only if $x^3 + 2x^2 + x + 1$ is
			irreducible in $\mathbb{Z}_3[x]$. Since $x^3 + 2x^2 + x + 1$ is a cubic function, if it
			does reduce it must factor into a quadratic term and a linear term. By Corollary 4.19 
			this is equivalent to saying that $x^3 + 2x^2 + x + 1$ is irreducible if and only if 
			$x^3 + 2x^2 + x + 1$ has no roots. The possible roots in $\mathbb{Z}_3[x]$ are $[0]$, 
			$[1]$, and $[2]$.

			Observe that
			\begin{align*}
				[0]^3 + 2[0]^2 + [0] + 1 & = [1] \\
				[1]^3 + 2[1]^2 + [1] + 1 & = [2] \\
				[2]^3 + 2[2]^2 + [2] + 1 & = [1]
			\end{align*}
			so we can conclude that $x^3 + 2x^2 + x + 1$ is irreducible and thus that 
			$\mathbb{Z}_3[x]/<x^3 + 2x^2 + x + 1>$ is a field.
		
		% ======================================================================================== %
		\item [5.3.1.b. ] \textbf{Determine whether $\mathbb{Z}_5[x]/<2x^3 - 4x^2 + 2x + 1>$ is a 
			field. Justify your answer.}

			By the same logic as the previous problem, we must check if $2x^3 - 4x^2 + 2x + 1$ is
			irreducible in $\mathbb{Z}_5[x]$ to see if $\mathbb{Z}_5[x]/<2x^3 - 4x^2 + 2x + 1>$ is a 
			field. Since $2x^3 - 4x^2 + 2x + 1$ is cubic, if it's reducible it must factor into a 
			quadratic term and a linear term, or in other words it must have a root. In 
			$\mathbb{Z}_5[x]$, the possible roots are $[0]$, $[1]$, $[2]$, $[3]$, and $[4]$.

			Observe that
			\begin{align*}
				2[0]^3 - 4[0]^2 + 2[0] + [1] & = [1] \\
				2[1]^3 - 4[1]^2 + 2[1] + [1] & = [1] \\
				2[2]^3 - 4[2]^2 + 2[2] + [1] & = [0]
			\end{align*}
			We see that $[2]$ is a root, so $2x^3 - 4x^2 + 2x + 1$ is reducible in $\mathbb{Z}_5[x]$
			and $\mathbb{Z}_5[x]/<2x^3 - 4x^2 + 2x + 1>$ is not a field.
		
		% ======================================================================================== %
		\item [5.3.1.c. ] \textbf{Determine whether $\mathbb{Z}_2[x]/<x^4 + x^2 + 1>$ is a 
			field. Justify your answer.}

			By the same logic above, we must check to see if $x^4 + x^2 + 1$ factors. Since the 
			equation has no roots, if it does factor it must factor into the product of two 
			quadratics. The only quadratic terms in $\mathbb{Z}_2[x]$ are $x^2$, $x^2 + 1$, $x^2 + 
			x$, and $x^2 + x + 1$.

			Observe that
			\begin{align*}
				(x^2 + 1)(x^2 + 1) = x^4 + x^2 + x^2 + 1
				(x^2 + 1)(x^2 + x + 1) = x^4 + x^3 + x^2 + x^2 + x + 1
			\end{align*}

		% ======================================================================================== %
		\item [5.3.5.a. ] \textbf{Verify that $\mathbb{Q}(\sqrt{3}) = \lbrace r + s\sqrt{3} \mid 
			r, s \in \mathbb{Q} \rbrace$ is a subfield of $\mathbb{R}$.}

			To show that $\mathbb{Q}(\sqrt{3})$ is a subfield of $\mathbb{R}$, we must show that
			$\mathbb{Q}(\sqrt{3})$ is closed under the subtraction and multiplication rules of 
			$\mathbb{R}$. 

			Consider the case of subtraction, given two arbitrary elements of $\mathbb{Q}(\sqrt{3})$
			\begin{align*}
				(a + b\sqrt{3}) - (c + d\sqrt{3}) & = a + b\sqrt{3} - c - d\sqrt{3} \qquad a, b, c,
					d \in \mathbb{Q} \\
					& = (a - c) + (b - d)\sqrt{3} 
			\end{align*}
			Therefore $\mathbb{Q}(\sqrt{3})$ is closed under subtraction.

			Now consider multiplication, again with arbitrary elements.
			\begin{align*}
				(a + b\sqrt{3}) * (c + d\sqrt{3}) & = ac + ad\sqrt{3} + bc\sqrt{3} + bd(3) \\
					& = (ac + 3bd) + (ad + bc)\sqrt{3}
			\end{align*}
			and $\mathbb{Q}(\sqrt{3})$ is closed under multiplication.

			Since $\mathbb{Q}(\sqrt{3})$ is closed under subtraction and multiplication, it is a 
			subring of $\mathbb{R}$.

		% ======================================================================================== %
		\item [5.3.5.b. ] \textbf{Show that $\mathbb{Q}(\sqrt{3})$ is isomorphic to $\mathbb{Q}[x]/
			<x^2 - 3>$.}

		% ======================================================================================== %
		\item [5.3.10. ] \textbf{Show that $\mathbb{Q}[x]/<x^2 - 2>$ is not isomorphic to 
			$\mathbb{Q}[x]/<x^2 - 3>$.}

			Since $\mathbb{Q}[x]/<x^2 - 2>$ is isomorphic to $\mathbb{Q}(\sqrt{2})$ and 
			$\mathbb{Q}[x]/<x^2 - 3>$ is isomorphic to $\mathbb{Q}(\sqrt{3})$, we can see that if
			$\mathbb{Q}[x]/<x^2 - 2>$ is isomorphic to $\mathbb{Q}[x]/<x^2 - 3>$ it must be that 
			$\mathbb{Q}(\sqrt{2})$ is isomorphic to $\mathbb{Q}(\sqrt{3})$. As shown in class, this
			is not true by the following:

			Let $f$ be an isomorphism from $\mathbb{Q}(\sqrt{2})$ to $\mathbb{Q}(\sqrt{3})$. Then 
			we know that
			\begin{align*}
				f(\sqrt{2}) & = r + s\sqrt{3} \qquad r, s \in \mathbb{Q} \\
				f(2) & = f(1 + 1) = f(1) + f(1) \\
				f(1) & = 1 \\
				f(2) & = 2 \\
				f(2) & = f(\sqrt{2} * \sqrt{2}) = f(\sqrt{2}) * f(\sqrt{2}) \\
			 		 & = (r + s\sqrt{3})(r + s\sqrt{3}) 
			\end{align*}
			Then it must be that
			\begin{align*}
				2 & = (r +s\sqrt{3})2 \\
					& = r2 + 3s2 + 2rs\sqrt{3} \\ 
				2 + 0\sqrt{3} & = r2 + 3s2 + 2rs\sqrt{3} 
			\end{align*}
			And
			\begin{align*}
				2 = r2 + 3s2 \\
				0 = 2rs			
			\end{align*}
			Which is a contradiction.

		% ======================================================================================== %
	\end{enumerate}
\end{document}

