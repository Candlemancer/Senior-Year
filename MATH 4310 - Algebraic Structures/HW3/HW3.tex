\documentclass{article}
\usepackage{amsmath}
\usepackage{amssymb}
\usepackage{enumerate}
\title{Homework 3 \\ Section 1.3}
\author{Jonathan Petersen \\ A01236750}
\date{January 27th, 2016}
\begin{document}
	\maketitle
	\hrule
	\begin{enumerate}
		% ==========================================================================================
		\item[7.]	\textbf{If $a, b, c \in \mathbb{Z}$ and $p$ is a prime such that $p | a$ and
					$p | a + bc$, prove that $p | b$ or $p | c$.}

					Because $p | a$ we know that 

					\begin{equation}
						a = pk \qquad k \in \mathbb{Z}				
					\end{equation}

					By the same logic, we can see that 

					\begin{equation}
						a + bc = pl \qquad l \in \mathbb{Z}
					\end{equation}

					If we substitute for $a$ in equation 2, we find that

					\begin{align*}
						pk + bc &= pl \\
						bc &= pl - pk \\
						bc &= p(l - k) \\
						bc &= pm \qquad m = l - k \therefore m \in \mathbb{Z}
					\end{align*}

					Therefore $p | bc$. Finally, by Theorem 1.5, it must be that $p | b$ or $p | c$ 
					$_{\blacksquare}$

		% ==========================================================================================
		\item[15.]	\textbf{If $p$ is prime and $p | a^n$, is it true that $p^n | a^n$? Justify your 
					answer.}

					Given a prime number $p$ such that $p | a^n$ for some $n \in \mathbb{Z}$, then 
					by Collorary 1.6 it must be that $p | a$. It then follows that:

					\begin{align*}
						a &= kp \qquad k \in \mathbb{Z} \\
						a^n &= (kp)^n \\
						a^n &= lp^n \qquad l = k^n \therefore l \in \mathbb{Z} \\
					\end{align*}

					Which, by definition, means that $p^n | a^n$ $_{\blacksquare}$

		% ==========================================================================================
		\item[21.]	\textbf{If $c^2 = ab$ and $gcd(a, b) = 1$, prove that $a, b$ are perfect 
					squares.} 

					By the funamental theorem of arithmetic, it must be that

					\begin{align*}
						a &= q_1^{u_1} * q_2^{u_2} * ... * q_n^{u_n}  
							\qquad q \text{ is prime} \qquad u \in \mathbb{Z}\\
						b &= r_1^{v_1} * r_2^{v_2} * ... * r_n^{v_n} 
							\qquad r \text{ is prime} \qquad v \in \mathbb{Z}\\
						c &= p_1^{t_1} * p_2^{t_2} * ... * p_n^{t_n} 
							\qquad p \text{ is prime} \qquad t \in \mathbb{Z}\\
						c^2 &= p_1^{2t_1} * p_2^{2t_2} * ... * p_n^{2t_n} \\
						ab = c^2 &= p_1^{2t_1} * p_2^{2t_2} * ... * p_n^{2t_n}
					\end{align*}

					Furthermore, since $gcd(a, b) = 1$, we know that $a$ and $b$ have no factors in 
					common greater than 1, including prime factors. Therefore, the prime factors of 
					$a$ and $b$ individually must form a basis or partition of the factors of $ab$, 
					with no overlapping factors. Or, put another way, that $a$ and $b$ can both be
					expressed using some subset of the prime factors of $ab$, with both subsets 
					being disjoint.

					Observe now that every factor in the prime factorization of $ab$ is a perfect 
					square number. By the properties of multiplication, any product of square
					numbers must also be a square number, therefore no matter the factors of $ab$
					are partioned into the factors of $a$ and $b$, $a$ and $b$ must be square
					numbers themselves $_{\blacksquare}$

		% ==========================================================================================
		\item[25.]	\textbf{Let $p$ be prime and $1 \leq k < p$. Prove that $p$ divides the binomial
					coefficient $\binom{p}{k}$.}

					Given $p, k$ as described in the statement, observe that 

					\begin{align*}
						\binom{p}{k} &= \frac{p!}{k!(p-k)!}
					\end{align*}

					Because $1 \leq k < p$, we know that none of the terms in the denominator will 
					ever cancel with the factor of $p$ in the numerator. $k!$ cannot cancel with it 
					because $k < p$, and $(p-k)!$ cannot cancel with it because $k > 0$, 
					so $(p - k) < p$. Therefore we can rewrite the equation as

					\begin{align*}
						\binom{p}{k} &= \frac{p(p-1)!}{k!(p-k)!} \\
							&= p\frac{(p-1)!}{k!(p-k)!} \\
							&= pn \qquad n = \tfrac{(p-1)!}{k!(p-k)!} \therefore n \in \mathbb{Z}
					\end{align*}	

					Which, by definition, means that $p | \binom{p}{k}$ $_{\blacksquare}$

		% ==========================================================================================
		\item[26.]	\textbf{If $n$ is a positive integer, prove that there exist $n$ consecutive 
					composite integers.}

					Let 

					\begin{equation*}
						S = \lbrace{}(n+1)! + 0, (n+1)! + 1, (n+1) + 2, ..., (n+1)! + k\rbrace{} \\
							\qquad n, k \in \mathbb{Z} \qquad k = n - 1
					\end{equation*}

					It is obvious that the elements of $S$ must be consecutive integers, and that
					$S$ must contain $k + 1 = n$ entries, so all that remains is to prove that every
					element in $S$ is composite.

					Let us consider an arbitrary element of $S$, 

					\begin{align*}
						s &= (n+1)! + i \qquad 0 \leq i < n \qquad i \in \mathbb{Z} \\
						  &= (n * n-1 * n-2 * ... * i * ... * 2 * 1) + i \\
						  &= ij + i \qquad ij = (n+1)! \therefore j \in \mathbb{Z} \\
						  &= i(j + 1) \\
					\end{align*}

					And since addition is closed over the integers, $j + 1 \in \mathbb{Z}$, so 
					$i | s$ for all $s$. This means that $s$ is composite, and by extension $S$ is a
					collection of $n$ consecutive composite numbers $_{\blacksquare}$

		% ==========================================================================================
		\item[31.]	\textbf{If $p$ is a positive prime, prove that $\sqrt{p}$ is irrational.}

					We will prove by contradiction. That is, hypothesize that there exists $p$ such 
					that:

					\begin{align*}
						\sqrt{p} &= \frac{a}{b} \qquad a, b \in \mathbb{Z} \qquad gcd(a, b) = 1 \\
						p &= (\frac{a}{b})^2
					\end{align*}

					We immediately see that 

					\begin{align*}
						p = (\frac{a}{b})^2 \implies b^2p &= a^2 \\
						kp &= a^2 \qquad k = b^2 \therefore k \in \mathbb{Z}
					\end{align*}

					Which by definition means that $p | a^2$, and by Theorem 1.5 means that $p | a$.
					Using this fact, it follows that

					\begin{align*}
						b^2p &= a^2 \\
						b^2p &= (pk)^2 \\
						b^2p &= p^2k^2 \\
						b^2 &= pk^2 \\
						b^2 &= pl \qquad l = k^2 \therefore k \in \mathbb{Z}
					\end{align*}

					Which again means that $p | b^2$ and therefore $p | b$. However, this is a 
					contradiciton! If $p | a$ and $p | b$, then $gcd(a, b) \neq 1$, contrary to the 
					given hypothesis. Therefore, the hypothesis must be false, and it must needs be 
					that if $p$ is a positive prime then $\sqrt{p}$ is irrational. 
					$_{\blacksquare}$

	\end{enumerate}
\end{document}

