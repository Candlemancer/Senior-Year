\documentclass{article}
\usepackage{amsmath}
\usepackage{amssymb}
\usepackage{enumerate}
\usepackage{dcolumn}
\newcolumntype{2}{D{.}{5.0pt}{2.0}}
\title{Homework 6 \\ Section 3.1}
\author{Jonathan Petersen \\ A01236750}
\date{February 29th, 2016}
\begin{document}
	\maketitle
	\hrule 
	\vspace{5mm}
	\begin{enumerate}
		% ======================================================================================== %
		\item [3.] \textbf{Let $F = \lbrace 0, e, a, b \rbrace$ with the operations given in the 
					text. Assume associativity and distributivity, and show that $F$ is a field.}

			To show that $F$ is a field, we must prove nine things about $F$. Namely that
			\begin{enumerate}
				\item [(a)]$F$ is closed under addition.
				\item [(b)]$F$ has associative addition.
				\item [(c)]$F$ has commutative addition.
				\item [(d)]$F$ has an additive identity.
				\item [(e)]$F$ has an additive inverse for every element.
				\item [(f)]$F$ is closed under multiplication.
				\item [(g)]$F$ has associative multiplication.
				\item [(h)]$F$ has distributive laws.
				\item [(i)]$F$ has a multiplicative inverse for every element.
			\end{enumerate}

			And indeed, as follows:

			\begin{enumerate}
				\item [Property (a):] Since every element in the addition table for $F$ is itself an 
					element of $F$, we can see by exhaustive inspection that $F$ is closed under
					addition.
				\item [Property (b):] Given by assumption.
				\item [Property (c):] Again, by exhaustive inspection of the addition table of $F$ we 
					see that the table is symmetric along the main diagonal (i.e. for any element
					$a_{i,j}$ in row $i$ and column $j$ of the addition table, $a_{i,j} = a_{j,i}$).
					This implies commutativity of addition on $F$.
				\item [Property (d):] From the addition table:
					\begin{align*}
						0 + 0 & = 0 \\
						e + 0 & = e \\
						a + 0 & = a \\
						b + 0 & = b
					\end{align*}
					Therefore 0 is the additive identity of $F$.
				\item [Property (e):] From the addition table:
					\begin{align*}
						0 + 0 & = 0 \\
						e + e & = 0 \\
						a + a & = 0 \\
						b + b & = 0
					\end{align*}
					And so by exhaustive inspection each element of $F$ has an additive inverse.
				\item [Property (f):] Similar to Property (a), we note that every element of the
					multiplication table of $F$ is itself an element of $F$, and so $F$ must be 
					closed under multiplication.
				\item [Property (g):] Given by assumption.
				\item [Property (h):] Given by assumption.
				\item [Property (i):] To show that each nonzero element of $F$ has a multiplicative 
					inverse, we must first define the identity of $F$. From the multiplication 
					table:
					\begin{align*}
						0 * e & = 0 \\
						e * e & = e \\
						a * e & = a \\
						b * e & = b \\
					\end{align*}
					So $e$ must be an identity of $F$. Now, to show that every nonzero element in 
					$F$ has a multiplicative inverse:
					\begin{align*}
						e * e & = e \\
						a * b & = e \\
						b * a & = e \\
					\end{align*}
					and since $0$ is the zero element of $F$, we can see that indeed, every nonzero
					element of $F$ has a multiplicative inverse.
			\end{enumerate}

			Since all of the above hold, it is apparent that $F$ must be a field $_{\blacksquare}$

		% ======================================================================================== %
		\item [5.b.] \textbf{Is a matrix of the form $\begin{bmatrix} a & b \\ 0 & c \end{bmatrix}$ 
						with $a, b, c \in \mathbb{Z}$ a subring of $M(\mathbb{R})$? \\
						Does it have an identity?}

			Let $A = \begin{bmatrix} a & b \\ 0 & c \end{bmatrix}$ with $a, b, c \in \mathbb{Z}$. To
			show that the set of all $A$ is a subring of $M(\mathbb{R})$ we must prove the following
			five things:

			\begin{enumerate}
				\item [(a)] The set of all $A$ is a subset of $M(\mathbb{R})$.
				\item [(b)] The set of all $A$ is closed under addition.
				\item [(c)] The set of all $A$ is closed under multiplication.
				\item [(d)] The zero element of $M(\mathbb{R})$ is in the set of all $A$.
				\item [(e)] Every $A$ has an additive inverse.
			\end{enumerate}

			Remark: As discussed in class, observe that if the set of all $A$ is closed under 
			subtraction, it must be that properties (b), (d), and (e) are true. We will not use this
			fact in this problem, but subsequent problems may take advantage of the fact.

			Indeed:

			\begin{enumerate}
				\item [Property (a):] Since $\mathbb{Z} \subseteq \mathbb{R}$ and $0 \in \mathbb{R}$, 
					and since $A$ is defined to be composed entirely of elements of $\mathbb{Z}$, it
					holds true that the set of all $A$ is a subset of the set of all two-by-two 
					matrices composed entirely of real numbers, namely $M(\mathbb{R})$.

				\item [Property (b):] Given $A_1 = \begin{bmatrix} a & b \\ 0 & c \end{bmatrix}$ and 
					$A_2 = \begin{bmatrix} r & s \\ 0 & t \end{bmatrix}$ with $a, b, c, r, s, t \in
					\mathbb{Z}$, we find
					\begin{align*}
						A_1 + A_2 & = \begin{bmatrix} a & b \\ 0 & c \end{bmatrix} 
									+ \begin{bmatrix} r & s \\ 0 & t \end{bmatrix} \\
								  & = \begin{bmatrix} a + r & b + s \\ 0 & c + t \end{bmatrix} 
					\end{align*}
					and since $\mathbb{Z}$ is closed under addition, it must be that $a + r$, 
					$b + s$, $0$, and $c + t$ are all integers, and thus $A_1 + A_2$ is in the 
					set of all $A$ and the set of all $A$ is closed under addition. 

				\item [Property (c):] Given the same setup of $A_1$ and $A_2$ as above, we see that
					\begin{align*}
						A_1 * A_2 & = \begin{bmatrix} a & b \\ 0 & c \end{bmatrix} 
									* \begin{bmatrix} r & s \\ 0 & t \end{bmatrix} \\
								  & = \begin{bmatrix} ar & as + bt \\ 0 & ct \end{bmatrix} 
					\end{align*}
					which again uses the fact that $\mathbb{Z}$ is closed under multiplication to 
					show that the set of all $A$ is also closed under multiplication.

				\item [Property (d):] The zero element of $M(\mathbb{R})$ is 
					$\begin{bmatrix}0 & 0 \\ 0 & 0 \end{bmatrix}$, which is an element of the set of
					all $A$ when $a, b, c = 0$. 

				\item [Property (e):] Since every element $z \in \mathbb{Z}$ has an additive inverse
					$-z$, we can see that the additive inverse for any 
					$A_1 = \begin{bmatrix} a & b \\ 0 & c \end{bmatrix}$ must be
					$A_2 = \begin{bmatrix} -a & -b \\ 0 & -c \end{bmatrix}$, leading to 
					$A_1 + A_2 = \begin{bmatrix} 0 & 0 \\ 0 & 0 \end{bmatrix}$, the zero element of 
					the set of all $A$.
			\end{enumerate}

			Also, the standard identity of $M(\mathbb{R})$, 
			$I = \begin{bmatrix} 1 & 0 \\ 0 & 1 \end{bmatrix}$, is also in the set of all $A$, when 
			$a, c = 1$ and $b = 0$. Furthermore, it is also an identity of the set of all $A$, since
			\begin{align*}
				A * I & = \begin{bmatrix} a & b \\ 0 & c \end{bmatrix} 
						* \begin{bmatrix} 1 & 0 \\ 0 & 1 \end{bmatrix} \\
					  & = \begin{bmatrix} a & b \\ 0 & c \end{bmatrix} \\
					  & = A
			\end{align*}

			Therefore, the set of all $A$ is a subring of $M(\mathbb{R})$ with an identity 
			$_{\blacksquare}$
		% ======================================================================================== %
		\item [5.c.] \textbf{Is a matrix of the form $\begin{bmatrix} a & b \\ c & 0 \end{bmatrix}$ 
						with $a, b, c \in \mathbb{R}$ a subring of $M(\mathbb{R})$? \\
						Does it have an identity?}

			Let $A = \begin{bmatrix} a & b \\ c & 0 \end{bmatrix} \qquad a, b, c \in \mathbb{R}$.
			As discussed in the previous problem, we can show that the set of all $A$ can be shown
			to be a subring of $M(\mathbb{R})$ by proving the following points:

			\begin{enumerate}
				\item [(a)] The set of all $A$ is a subset of $M(\mathbb{R})$.
				\item [(b)] The set of all $A$ is closed under subtraction.
				\item [(c)] The set of all $A$ is closed under multiplication.
			\end{enumerate}

			Indeed:

			\begin{enumerate}
				\item [Property (a):] By the same logic as the previous problem, since $a, b, c \in 
					\mathbb{R}$ and $\mathbb{R} \subseteq \mathbb{R}$ the set of all 
					$A \subseteq M(\mathbb{R})$.
				\item [Property (b):] Given $A_1 = \begin{bmatrix} a & b \\ c & 0 \end{bmatrix}$ and 
					$A_2 = \begin{bmatrix} r & s \\ t & 0 \end{bmatrix}$ with $a, b, c, r, s, t \in
					\mathbb{R}$, we find
					\begin{align*}
						A_1 - A_2 & = \begin{bmatrix} a & b \\ c & 0 \end{bmatrix} 
									+ \begin{bmatrix} r & s \\ t & 0 \end{bmatrix} \\
								  & = \begin{bmatrix} a - r & b - s \\ 0 & c - t \end{bmatrix} 
					\end{align*}
					and since $\mathbb{R}$ is closed under subtraction, it must be that $a - r$, 
					$b - s$, $0$, and $c - t$ are all real numbers, and thus $A_1 - A_2$ is in the 
					set of all $A$ and the set of all $A$ is closed under subtraction. 
				\item [Property (c):] Given the same setup of $A_1$ and $A_2$ as above, we see that
					\begin{align*}
						A_1 * A_2 & = \begin{bmatrix} a & b \\ c & 0 \end{bmatrix} 
									* \begin{bmatrix} r & s \\ t & 0 \end{bmatrix} \\
								  & = \begin{bmatrix} ar & as + bt \\ 0 & ct \end{bmatrix} 
					\end{align*}
					which again uses the fact that $\mathbb{R}$ is closed under multiplication to 
					show that the set of all $A$ is also closed under multiplication.
			\end{enumerate}

			However, this time there is no way to form the identity $I$, since $A$ has a $0$ on the 
			main diagonal. Therefore, the set of all $A$ is a subring of $M(\mathbb{R})$, but 
			without an identity $_{\blacksquare}$

		% ======================================================================================== %
		\item [5.d.] \textbf{Is a matrix of the form $\begin{bmatrix} a & 0 \\ a & 0 \end{bmatrix}$ 
						with $a \in \mathbb{R}$ a subring of $M(\mathbb{R})$? \\
						Does it have an identity?}

			Let $A = \begin{bmatrix} a & 0 \\ a & 0 \end{bmatrix} \qquad a \in \mathbb{R}$. It then 
			follows that:

			\begin{enumerate}
				\item [Property (a):] By the same logic as the previous problem, since $a, b, c \in 
					\mathbb{R}$ and $\mathbb{R} \subseteq \mathbb{R}$ the set of all 
					$A \subseteq M(\mathbb{R})$.
				\item [Property (b):] Given $A_1 = \begin{bmatrix} a & 0 \\ a & 0 \end{bmatrix}$ and 
					$A_2 = \begin{bmatrix} b & 0 \\ b & 0 \end{bmatrix}$ with $a, b \in \mathbb{R}$,
					we find
					\begin{align*}
						A_1 - A_2 & = \begin{bmatrix} a & 0 \\ a & 0 \end{bmatrix} 
									+ \begin{bmatrix} b & 0 \\ b & 0 \end{bmatrix} \\
								  & = \begin{bmatrix} a - b & 0 \\ a - b & 0 \end{bmatrix} 
					\end{align*}
					and since $\mathbb{R}$ is closed under subtraction, it must be that $a - b$ is a
					real number, and thus $A_1 - A_2$ is in the set of all $A$ and the set of all 
					$A$ is closed under subtraction. 
				\item [Property (c):] Given the same setup of $A_1$ and $A_2$ as above, we see that
					\begin{align*}
						A_1 * A_2 & = \begin{bmatrix} a & 0 \\ a & 0 \end{bmatrix} 
									+ \begin{bmatrix} b & 0 \\ b & 0 \end{bmatrix} \\
								  & = \begin{bmatrix} ab & 0 \\ ab & 0 \end{bmatrix} 
					\end{align*}
					which again uses the fact that $\mathbb{R}$ is closed under multiplication to 
					show that the set of all $A$ is also closed under multiplication.
			\end{enumerate}

			Again, this time there is no way to form the identity $I$, since $A$ has a $0$ on the 
			main diagonal. Therefore, the set of all $A$ is a subring of $M(\mathbb{R})$, but 
			without an identity $_{\blacksquare}$

		% ======================================================================================== %
		\item [6.a.] \textbf{Show that the set $R$ of all multiples of 3 is a subring of 
						$\mathbb{Z}$.}

			Given:
			\begin{equation*}
				R = \lbrace r = 3x | x \in \mathbb{Z} \rbrace
			\end{equation*}
			we may show that $R$ is a subring of $\mathbb{Z}$ using the simplified method discussed
			in problem 5.

			\begin{enumerate}
				\item [Property (a):] Since $\mathbb{Z}$ is closed under multiplication, it is 
					trivial to see that 
					\begin{equation*}
						r = 3x | x \in \mathbb{Z} \implies r \in \mathbb{Z}
					\end{equation*} 
					and so $R \subseteq \mathbb{Z}$.
				\item [Property (b):] Let $r_1 = 3m$, $r_2 = 3n$ for some $m, n \in \mathbb{Z}$. 
					Then 
					\begin{align*}
						r_1 - r_2 & = 3m - 3n \\
								  & = 3(m - n) \\
								  & = 3k \qquad \qquad k = m - n
					\end{align*}
					and since $\mathbb{Z}$ is closed under subtraction, $k \in \mathbb{Z}$ and 
					$r_1 - r_2 \in R$.
				\item [Property (c):] Given $r_1$, $r_2$ as above,
					\begin{align*}
						r_1 * r_2 & = 3m * 3n \\
								  & = 3(3mn) \\
								  & = 3k \qquad \qquad k = 3mn \implies k \in \mathbb{Z}
					\end{align*}
					so again, $R$ is closed under multiplication.
			\end{enumerate}

			Therefore, $R$ is a subring of $\mathbb{Z}$ $_{\blacksquare}$

		% ======================================================================================== %
		\item [6.b.] \textbf{Let $k$ be a fixed integer. Show that the set of all multiples of $k$ 
						is a subring of $\mathbb{Z}$.}		

			Given:
			\begin{equation*}
				R = \lbrace r = kx | x, k \in \mathbb{Z} \rbrace
			\end{equation*}
			we may show that $R$ is a subring of $\mathbb{Z}$ using the simplified method discussed
			in problem 5.

			\begin{enumerate}
				\item [Property (a):] Since $\mathbb{Z}$ is closed under multiplication, it is 
					trivial to see that 
					\begin{equation*}
						r = kx | x, k \in \mathbb{Z} \implies r \in \mathbb{Z}
					\end{equation*} 
					and so $R \subseteq \mathbb{Z}$.
				\item [Property (b):] Let $r_1 = km$, $r_2 = kn$ for some $m, n, k \in \mathbb{Z}$. 
					Then 
					\begin{align*}
						r_1 - r_2 & = km - kn \\
								  & = k(m - n) \\
								  & = ki \qquad \qquad i = m - n \implies i \in \mathbb{Z}
					\end{align*}
					so $R$ is closed under subtraction.
				\item [Property (c):] Given $r_1$, $r_2$ as above,
					\begin{align*}
						r_1 * r_2 & = km * kn \\
								  & = k(kmn) \\
								  & = ki \qquad \qquad i = kmn \implies i \in \mathbb{Z}
					\end{align*}
					so again, $R$ is closed under multiplication.
			\end{enumerate}

			Therefore, $R$ is a subring of $\mathbb{Z}$ $_{\blacksquare}$

		% ======================================================================================== %
		\item [9.a.] \textbf{Let $R$ be a ring and consider the subset $R^*$ of $R \times R$ defined
						by $R^* = \lbrace (r, r) | r \in R \rbrace$.}
					 \textbf{If $R = \mathbb{Z}_6$, list the elements of $R^*$.}

					 \begin{equation*}
					 	R^* = \lbrace ([0], [0]), ([1], [1]), ([2], [2]), ([3], [3]), ([4], [4]), 
					 				  ([5], [5]) \rbrace
					 \end{equation*}

		% ======================================================================================== %
		\item [9.b.] \textbf{For any ring $R$, show that $R^*$ is a subring of $R \times R$.}

			Following the simplified proof:
			\begin{enumerate}
				\item [Property (a):] By definition, every element of $R^*$ is an element of 
					$R \times R$. Therefore $R^* \subseteq R \times R$.
				\item [Property (b):] Let $r_1 = (a, a)$, $r_2 = (b, b)$ for some $a, b \in R$. 
					Then 
					\begin{align*}
						r_1 - r_2 & = (a - b, a - b) \\
								  & = (c, c) \qquad \qquad c = a + b
					\end{align*}
					and since $R$ is a ring, $c = a + b \implies c \in R$, so $R^*$ is closed under 
					subtraction.
				\item [Property (c):] Given $r_1$, $r_2$ as above,
					\begin{align*}
						r_1 * r_2 & = (ab, ab) \\
								  & = (c, c) \qquad \qquad c = ab \implies c \in R
					\end{align*}
					so again, $R^*$ is closed under multiplication.
			\end{enumerate}

			Therefore, $R^*$ is a subring of $R \times R$ $_{\blacksquare}$

		% ======================================================================================== %
		\item [10.] \textbf{Is $S = \lbrace (a, b) | a + b = 0 \rbrace$ a subring of $\mathbb{Z} 
						\times \mathbb{Z}$? Justify your answer.}

			Consider closure under multiplication. Let $s_1 = (a, b)$, $s_2 = (c, d)$ for some 
			$a, b, c, d \in \mathbb{Z}$ such that $a + b = 0$ and $c + d = 0$. Observe that 
			$a + b = 0$ implies that $b = -a$ and likewise that $d = -c$. So:
			
			\begin{align*}
				s_1 * s_2 & = (ac, bd) \\
				ac + bd & = ac + (-a)(-c) \\
						& = ac + ac \\
						& \neq 0
			\end{align*}
			
			Or, in other words, $S$ is not closed under multiplication. Therefore, $S$ cannot be a 
			subring of $\mathbb{Z} \times \mathbb{Z}$ $_{\blacksquare}$

		% ======================================================================================== %
		\item [13.] \textbf{Let $\mathbb{Z}[\sqrt{2}]$ denote the set $\lbrace a + b\sqrt{2} | a, b 
						\in \mathbb{Z} \rbrace$. Show that $\mathbb{Z}[\sqrt{2}]$ is a subring of 
						$\mathbb{R}$.}

			Following the simplified proof:
			\begin{enumerate}
				\item [Property (a):] Since $a + b\sqrt{2} \in \mathbb{R}$ for any $a, b \in 
					\mathbb{Z}$, it must be that $\mathbb{Z}[\sqrt{2}] \subseteq \mathbb{R}$.
				\item [Property (b):] Let $z_1 = a + b\sqrt{2}$, $z_2 = c + d\sqrt{2}$ for some 
					$a, b, c, d \in \mathbb{Z}$. Then 
					\begin{align*}
						z_1 - z_2 & = a + b\sqrt{2} - (c + d\sqrt{2}) \\
								  & = a - c + b\sqrt{2} - d\sqrt{2} \\
								  & = (a - c) + (b - d)\sqrt{2} \\
								  & = m + n\sqrt{2} \qquad \qquad m = a - c, n = b - d
					\end{align*}
					and so $\mathbb{Z}[\sqrt{2}]$ is closed under subtraction.
				\item [Property (c):] Given $z_1$, $z_2$ as above,
					\begin{align*}
						z_1 * z_2 & = (a + b\sqrt{2})(c + d\sqrt{2}) \\
								  & = ac + ad\sqrt{2} + (b\sqrt{2})c + (b\sqrt{2})(d\sqrt{2}) \\
								  & = ac + ad\sqrt{2} + bc\sqrt{2} + 2bd \\
								  & = ac + 2bd + (ad + bc)\sqrt{2} \\
								  & = m + n\sqrt{2} \qquad \qquad m = ac + 2bd, n = ad + bc
					\end{align*}
					so again, $\mathbb{Z}[\sqrt{2}]$ is closed under multiplication.
			\end{enumerate}

			Therefore, $\mathbb{Z}[\sqrt{2}]$ is a subring of $\mathbb{R}$ $_{\blacksquare}$

		% ======================================================================================== %
		\pagebreak
		\item [15.a.] \textbf{Write out the addition and multiplication tables for $\mathbb{Z}_2 
						\times \mathbb{Z}_3$.}

			\begin{center}
				\renewcommand\arraystretch{1.3}
				\setlength\doublerulesep{0pt}
				\begin{tabular}{r || *{6}{c|} }
								 $+$ & $([0]_2, [0]_3)$ & $([0]_2, [1]_3)$ & $([0]_2, [2]_3)$ & 
								 $([1]_2, [0]_3)$ & $([1]_2, [1]_3)$ & $([1]_2, [2]_3)$ \\[0.3pt]
								 \hline\hline
					$([0]_2, [0]_3)$ & $([0]_2, [0]_3)$ & $([0]_2, [1]_3)$ & $([0]_2, [2]_3)$ & 
					$([1]_2, [0]_3)$ & $([1]_2, [1]_3)$ & $([1]_2, [2]_3)$ \\[0.3pt]\hline
					$([0]_2, [1]_3)$ & $([0]_2, [1]_3)$ & $([0]_2, [2]_3)$ & $([0]_2, [0]_3)$ & 
					$([1]_2, [1]_3)$ & $([1]_2, [2]_3)$ & $([1]_2, [0]_3)$ \\[0.3pt]\hline
					$([0]_2, [2]_3)$ & $([0]_2, [2]_3)$ & $([0]_2, [0]_3)$ & $([0]_2, [1]_3)$ & 
					$([1]_2, [2]_3)$ & $([1]_2, [0]_3)$ & $([1]_2, [1]_3)$ \\[0.3pt]\hline
					$([1]_2, [0]_3)$ & $([1]_2, [0]_3)$ & $([1]_2, [1]_3)$ & $([1]_2, [2]_3)$ & 
					$([0]_2, [0]_3)$ & $([0]_2, [1]_3)$ & $([0]_2, [2]_3)$ \\[0.3pt]\hline
					$([1]_2, [1]_3)$ & $([1]_2, [1]_3)$ & $([1]_2, [2]_3)$ & $([1]_2, [0]_3)$ & 
					$([0]_2, [1]_3)$ & $([0]_2, [2]_3)$ & $([0]_2, [0]_3)$ \\[0.3pt]\hline
					$([1]_2, [2]_3)$ & $([1]_2, [2]_3)$ & $([1]_2, [0]_3)$ & $([1]_2, [1]_3)$ & 
					$([0]_2, [2]_3)$ & $([0]_2, [0]_3)$ & $([0]_2, [1]_3)$ \\[0.3pt]\hline
				\end{tabular}
			\end{center}
			\begin{center}
				\renewcommand\arraystretch{1.3}
				\setlength\doublerulesep{0pt}
				\begin{tabular}{r || *{6}{c|} }
							$\times$ & $([0]_2, [0]_3)$ & $([0]_2, [1]_3)$ & $([0]_2, [2]_3)$ & 
							$([1]_2, [0]_3)$ & $([1]_2, [1]_3)$ & $([1]_2, [2]_3)$ \\[0.3pt]
							\hline\hline
					$([0]_2, [0]_3)$ & $([0]_2, [0]_3)$ & $([0]_2, [0]_3)$ & $([0]_2, [0]_3)$ & 
					$([0]_2, [0]_3)$ & $([0]_2, [0]_3)$ & $([0]_2, [0]_3)$ \\[0.3pt]\hline
					$([0]_2, [1]_3)$ & $([0]_2, [0]_3)$ & $([0]_2, [1]_3)$ & $([0]_2, [2]_3)$ & 
					$([0]_2, [0]_3)$ & $([0]_2, [1]_3)$ & $([0]_2, [2]_3)$ \\[0.3pt]\hline
					$([0]_2, [2]_3)$ & $([0]_2, [0]_3)$ & $([0]_2, [2]_3)$ & $([0]_2, [1]_3)$ & 
					$([0]_2, [0]_3)$ & $([0]_2, [2]_3)$ & $([0]_2, [1]_3)$ \\[0.3pt]\hline
					$([1]_2, [0]_3)$ & $([0]_2, [0]_3)$ & $([0]_2, [0]_3)$ & $([0]_2, [0]_3)$ & 
					$([1]_2, [0]_3)$ & $([1]_2, [0]_3)$ & $([1]_2, [0]_3)$ \\[0.3pt]\hline
					$([1]_2, [1]_3)$ & $([0]_2, [0]_3)$ & $([0]_2, [1]_3)$ & $([0]_2, [2]_3)$ & 
					$([1]_2, [0]_3)$ & $([1]_2, [1]_3)$ & $([1]_2, [2]_3)$ \\[0.3pt]\hline
					$([1]_2, [2]_3)$ & $([0]_2, [0]_3)$ & $([0]_2, [2]_3)$ & $([0]_2, [1]_3)$ & 
					$([1]_2, [0]_3)$ & $([1]_2, [2]_3)$ & $([1]_2, [1]_3)$ \\[0.3pt]\hline
				\end{tabular}
			\end{center}

		% ======================================================================================== %
		\item [16.a.] \textbf{Let 
						$A = \begin{bmatrix} 1 & 1 \\ 1 & 1 \end{bmatrix}$ and 
						0 $= \begin{bmatrix} 0 & 0 \\ 0 & 0 \end{bmatrix}$ in $M(\mathbb{R})$. Let 
						$S$ be the set of all matrices $B$ such that $AB = $ 0. List three matrices
						in $S$.}

			\begin{align*}
				S_1 & = \begin{bmatrix} 1 & -1 \\ -1 & 1 \end{bmatrix} \\
				S_2 & = \begin{bmatrix} 2 & -2 \\ -2 & 2 \end{bmatrix} \\
				S_3 & = \begin{bmatrix} 3 & -3 \\ -3 & 3 \end{bmatrix} \\
				S_a & = \begin{bmatrix} a & -a \\ -a & a \end{bmatrix} \qquad a \in \mathbb{R} \\
			\end{align*}
		% ======================================================================================== %
		\item [16.b.] \textbf{Prove that $S$ is a subring of $M(\mathbb{R})$.}

			Following the simplified proof:
			\begin{enumerate}
				\item [Property (a):] Assuming all of the entries of $S$ are real numbers, then it 
					is obvious that $S \subseteq M(\mathbb{R})$.
				\item [Property (b):] Using the generalization in the previous problem, let
					\begin{align*}
						S_a & = 
							\begin{bmatrix} a & -a \\ -a & a \end{bmatrix} \qquad a \in \mathbb{R}\\
						S_b & = 
							\begin{bmatrix} b & -b \\ -b & b \end{bmatrix} \qquad b \in \mathbb{R}\\
					\end{align*}
					Then
					\begin{align*}
						S_a - S_b & = 
							\begin{bmatrix} a - b & -a - -b \\ -a - -b & a - b \end{bmatrix} \\
								  & = 
							\begin{bmatrix} a - b & -(a - b) \\ -(a - b) & a - b \end{bmatrix} \\
								  & = 
							\begin{bmatrix} n & -n \\ -n & n \end{bmatrix} \qquad n = a - b \\
					\end{align*}
					and because $n \in \mathbb{R}$, we see that $S$ is closed under subtraction.
				\item [Property (c):] Given $S_a$, $S_b$ as above,
					\begin{align*}
						S_a * S_b & = \begin{bmatrix} ab + (-a)(-b) & a(-b) + (-a)b \\
									  				  (-a)b + a(-b) & (-a)(-b) + ab \end{bmatrix} \\
								  & = \begin{bmatrix} ab + ab & -ab + -ab \\
									  				  -ab + -ab & ab + ab \end{bmatrix} \\
  								  & = \begin{bmatrix} 2ab & -2ab \\ -2ab & ab \end{bmatrix} \\
  								  & = \begin{bmatrix} n & -n \\ -n & n \end{bmatrix} \qquad \qquad
  								  		n = 2ab \implies n \in \mathbb{R} \\
					\end{align*}
					so again, $S$ is closed under multiplication.
			\end{enumerate}

			Therefore, $S$ is a subring of $M(\mathbb{R})$ $_{\blacksquare}$

		% ======================================================================================== %
		\item [25.] \textbf{Define a new addition and multiplication on $\mathbb{Q}$ by }

					\begin{equation*}
						r \oplus s = r + s + 1 \qquad and \qquad r \otimes s = rs + r + s
					\end{equation*}

					\textbf{Prove that with these new operations $\mathbb{Q}$ is a commutative ring 
						with identity. Is it an integral domain?}

			Let $S$ denote $\mathbb{Q}$ with our new operations. To show that $S$ is a commutative 
			ring with identity and consider its status as an integral domain, we must consider the 
			following:

			\begin{enumerate}
				\item [(a)]$S$ is closed under addition.
				\item [(b)]$S$ has associative addition.
				\item [(c)]$S$ has commutative addition.
				\item [(d)]$S$ has an additive identity.
				\item [(e)]$S$ has an additive inverse for every element.
				\item [(f)]$S$ is closed under multiplication.
				\item [(g)]$S$ has associative multiplication.
				\item [(h)]$S$ has distributive laws.
				\item [(i)]$S$ has commutative multiplication.
				\item [(j)]$S$ has an identity.
				\item [(k)]$s_1 \neq 0$ and $s_2 \neq 0$ implies $s_1 \otimes s_2 \neq 0$.
			\end{enumerate}

			Let $a, b, c$ be arbitrary elements of $S$. Then we see:

			\begin{enumerate}
				\item [Property (a):] 
					\begin{equation}
						a \oplus b = a + b + 1 
					\end{equation}
					Since $\mathbb{Q}$ is closed under addition, this implies that $a + b + 1 \in 
					\mathbb{Q}$ and so $S$ is closed under addition.
				\item [Property (b):]
					\begin{align*}
						(a \oplus b) \oplus c & = (a + b + 1) + c + 1 \\
						a \oplus (b \oplus c) & = a + (b + c + 1) + 1 \\
					\end{align*}
					Since addition in $\mathbb{Q}$ is associative,
					\begin{align*}
						(a + b + 1) + c + 1 & = a + b + c + 2 \\
						a + (b + c + 1) + 1 & = a + b + c + 2 \\
						a + (b + c + 1) + 1 & = a + (b + c + 1) + 1
					\end{align*}
					and so $S$ has associative addition.
				\item [Property (c):]
					\begin{align*}
						a \oplus b & = a + b + 1 \\
						b \oplus a & = b + a + 1 \\
					\end{align*}
					Since addition in $\mathbb{Q}$ is commutative, $a + b + 1 = b + a + 1$ and so
					$S$ has commutative addition.
				\item [Property (d):]
					Consider $i$ such that $a \oplus i = a$. Then
					\begin{align*}
						a \oplus i & = a + i + 1 = a\\
						i & = a + -a + -1 \\
						i & = -1
					\end{align*}
					So $-1$ is the additive identity of $S$.
				\item [Property (e):]
					Consider $x$ such that $a \oplus x = 0$. Then
					\begin{align*}
						a \oplus x & = a + x + 1 = 0 \\
						x & = -a + -1
					\end{align*}
					Since $-a$ and $-1$ are both in $\mathbb{Q}$, and since $\mathbb{Q}$ is closed
					under addition, this means that $x \in \mathbb{Q}$ exists for any $a$ which in 
					turn implies that every element of $S$ has an additive inverse, namely $x$.
				\item [Property (f):]
					\begin{align*}
						a \otimes b & = ab + a + b \\
					\end{align*}
					Since $\mathbb{Q}$ is closed under both standard addition and multiplication, 
					this implies that $ab + a + b \in \mathbb{Q}$ and $S$ is closed under 
					multiplication as well.
				\item [Property (g):]
					\begin{align*}
						(a \otimes b) \otimes c & = (ab + a + b)c + (ab + a + b) + c \\
												& = abc + ac + bc + ab + a + b + c \\
						a \otimes (b \otimes c) & = a(bc + b + c) + a + (bc + b + c) \\
												& = abc + ab + ac + a + bc + b + c \\
												& = abc + ac + bc + ab + a + b + c \\
												& = (a \otimes b) \otimes c \\
					\end{align*}
					So $S$ has associatve multiplication.
				\item [Property (h):]
					\begin{align*}
						a \otimes (b \oplus c) & = a(b \oplus c) + a + (b \oplus c) \\
											   & = a(b + c + 1) + a + (b + c + 1) \\
											   & = ab + ac + a + a + b + c + 1 \\
											   & = (ab + a + b) + (ac + a + c) + 1 \\
											   & = (a \otimes b) \oplus (a \otimes c) \\
					\end{align*}
					So $S$ has distributive laws.
				\item [Property (i):]
					\begin{align*}
						a \otimes b & = ab + a + b \\
						b \otimes a & = ba + b + a \\
					\end{align*}
					Since $\mathbb{Q}$ has commutative addition and multiplication, 
					$ab + a + b = ba + b + a$ and so $S$ also has commutative multiplication.
				\item [Property (j):]
					Consider $i$ such that $a \otimes i = a$. Then
					\begin{align*}
						a \otimes i & = ai + a + i = a \\
						ai + i & = 0 \\
						i(a + 1) & = 0
					\end{align*}
					Since $\mathbb{Q}$ is an integral domain, this implies that either $i = 0$ or 
					$(a + 1) = 0$. So for an arbitrary element $a \in S$, the multiplicative
					identity of $a$ is $0$, unless $a = -1$, in which case any $i \in \mathbb{Q}$
					is an identity, including $i = 0$. Either way, $S$ always has a multiplicative
					identity. 
				\item [Property (k):]
					Consider the opposite:
					\begin{align*}
						ab + a + b & = 0 \qquad a, b \neq 0 \\
						ab + a = -b \\
						a(b + 1) = -b \\ 
						a = \frac{-b}{b + 1} \\					
					\end{align*}
					Since fractions are well defined and closed on $\mathbb{Q}$, this implies that 
					$a \in \mathbb{Q}$ and that $a \otimes b = 0$ when $a = \frac{-b}{b + 1}$, even
					if $a$ and $b$ are nonzero. Therefore, this property is false for $S$.
			\end{enumerate}

			Since properties (a) - (j) were true, it holds that $S$ is a commutative ring with 
			identity. But since property (k) did not hold for $S$, it is not an integral domain
			$_{\blacksquare}$

		% ======================================================================================== %
		\item [27.] \textbf{Let $S$ be the set of rational numbers written with an odd denominator.
						Prove that $S$ is a subring of $\mathbb{Q}$ but is not a field.}

			Following the simplified proof:
			\begin{enumerate}
				\item [Property (a):] Since $\frac{a}{2k + 1} \in \mathbb{Q}$ for any $a, k \in 
					\mathbb{Z}$, it must be that $S \subseteq \mathbb{Q}$.
				\item [Property (b):] Let $s_1 = \frac{a}{2m + 1}$, $s_2 = \frac{b}{2n + 1}$ for 
					some $a, b, m, n \in \mathbb{Z}$. Then 
					\begin{align*}
						s_1 - s_2 & = \frac{a}{2m + 1} - \frac{b}{2n + 1} \\
								  & = \frac{a(2n + 1) - b(2m + 1)}{(2m + 1)(2n + 1)} \\
								  & = \frac{a(2n + 1) - b(2m + 1)}{4mn + 2m + 2n + 1} \\
								  & = \frac{c}{2d + 1} \\
								c & = a(2n + 1) - b(2m + 1) \qquad d = 2mn + m + n
					\end{align*}
					Since $\mathbb{Q}$ is closed under multiplication and addition, 
					$c, d \in \mathbb{Q}$ and $S$ is closed under subtraction.
				\item [Property (c):] Given $s_1$, $s_2$ as above,
					\begin{align*}
						s_1 * s_2 & = \frac{a}{2m + 1} * \frac{b}{2n + 1} \\
								  & = \frac{ab}{(2m + 1)(2n + 1)} \\
								  & = \frac{ab}{2d + 1} \qquad \qquad d = 2mn + m + n
					\end{align*}
					so again, $S$ is closed under multiplication.
			\end{enumerate}

			Therefore, $S$ is a subring of $\mathbb{Q}$. To show that $S$ is not a field, we must 
			show that there does not exist a multiplicative inverse of $S$. That is, that the 
			equation

			\begin{align*}
				s_1 * s_2 = 1 \qquad \qquad s_1, s_2 \neq 0
			\end{align*}

			does not always have a solution. Indeed, consider the example when $s_1 = \frac{2}{3}$.
			The multiplicative inverse of $\frac{2}{3}$ in $\mathbb{Q}$ is $\frac{3}{2}$, but that
			number does not have an odd denominator, and as such cannot be an element of $S$. 
			Therefore, $S$ is not a field $_{\blacksquare}$

		% ======================================================================================== %
		\item [30.] \textbf{The addition table and part of a multiplication table for a four-element
						ring are given in the text. Complete the multiplication table.}		
		% ======================================================================================== %

		\begin{center}
			\renewcommand\arraystretch{1.3}
			\setlength\doublerulesep{0pt}
			\begin{tabular}{r||*{4}{2|}}
				$\times$ & $w$ & $x$ & $y$ & $z$ \\
				\hline\hline
				$w$ & $w$ & $w$ & $w$ & $w$ \\ 
				\hline
				$x$ & $w$ & $y$ & $w$ & $y$ \\ 
				\hline
				$y$ & $w$ & $w$ & $w$ & $w$ \\ 
				\hline
				$z$ & $w$ & $y$ & $w$ & $y$ \\ 
				\hline
			\end{tabular}
		\end{center}
	\end{enumerate}
\end{document}

